\documentclass{article}
\usepackage[version=3]{mhchem}
\usepackage[english]{}
\usepackage[margin=2cm]{geometry}
\usepackage{enumitem}
\usepackage{graphicx}
\usepackage{float}
\usepackage{amsmath}
\usepackage{dcolumn}
\usepackage{bm}
\usepackage{amssymb}
\usepackage{verbatim}
\usepackage{color}
\definecolor{mygreen}{RGB}{28,172,0} % color values Red, Green, Blue
\definecolor{mylilas}{RGB}{170,55,241}
\usepackage{setspace}
\usepackage{parskip}
\usepackage[space]{cite}
\usepackage{url}
\usepackage{subcaption}
\usepackage{textcomp}
\usepackage{multirow}% http://ctan.org/pkg/multirow
\usepackage{hhline}% http://ctan.org/pkg/hhline
\usepackage{icomma}
\usepackage{placeins}
\usepackage[caption=false,font=footnotesize]{subfig}
% define the title
\author{\textit {Christopher Poresky}}
\title{\textbf {NE 250: Homework 1}}
\begin{document}

\maketitle
\section*{Problem 1}

The fuel for a certain breeder reactor consists of pellets composed of mixed oxides, $UO_{2}$ and
$PuO_{2}$, with the $PuO_{2}$ comprising approximately 30 weight$~\%$ of the mixture. The uranium is
essentially all $^{238}U$, whereas the plutonium contains the following isotopes: $^{239}Pu~(69.7~wt\%)$,
$^{240}Pu~(21.8~wt\%)$, $^{241}Pu~(5.8~wt\%)$, and $^{242}Pu~(2.7~wt\%)$. Calculate the number of atoms of
each isotope per gram of the fuel.

\hrulefill

We start by tabulating atomic masses from \textit{atom.kaeri.re.kr}:
\begin{itemize}
\item $m(^{238}U) = 238.050788423~u$
\item $m(^{239}Pu) = 239.052163591~u$
\item $m(^{240}Pu) = 240.05381375~u$
\item $m(^{241}Pu) = 241.056851661~u$
\item $m(^{242}Pu) = 242.058742809~u$
\item $m(^{16}O) = 15.99491461957~u$
\end{itemize}

Assuming $1~g$ of fuel, we know that $UO_{2}$ is $70~wt\%$ of the fuel, so we say it has a mass of $0.7~g$. $1~mole$ of $UO_{2}$ is $238.05 + 2(15.99) = 270.04~g$. We therefore have $\frac{0.7~g}{270.04~g} = 0.00259~moles~UO_{2}$. Then,

$$0.00259~moles~UO_{2} \times \frac{6.022 \times 10^{23}~atoms}{mole} = \mathbf{1.56 \times 10^{21}~atoms~^{238}U~per~gram~of~fuel}.$$

Moving on, we know that $PuO_{2}$ is $30~wt\%$ of the fuel, so we say it has a mass of $0.3~g$. In order to calculate the molar mass of $Pu$ in this mixture, we find the sum of the ratio of each weight percentage divided by each isotope's molar mass:

$$\frac{1}{M(Pu)} = \sum_{i} \frac{w_{i}}{M_{i}}$$

and we calculate that $M(Pu) = 239.465800990855~u$. Then, we know that the mass of 1 mole of $PuO_{2}$ is $239.47 + 2(15.99) = 271.4556~u$. We therefore have $\frac{0.3~g}{271.46~g} = 0.00111~moles~PuO_{2}$. We then have

$$0.00111~moles~Pu \times 239.4658~g/mol = 0.2646~g~Pu$$

Since we know the weight percentages, we can calculate our mass for each isotope of Pu:

\begin{itemize}
\item $m(^{239}Pu) =  0.2646~g \times 0.697 = 0.1844~g$
\item $m(^{240}Pu) =  0.2646~g \times 0.218 = 0.05769~g$
\item $m(^{241}Pu) =  0.2646~g \times 0.058 = 0.01535~g$
\item $m(^{242}Pu) =  0.2646~g \times 0.027 = 0.007145~g$
\end{itemize}

and we calculate our atoms per gram as

\begin{itemize}
\item $N(^{239}Pu) =  \frac{0.1844~g}{239.052163591~g/mol}\times \frac{6.022 \times 10^{23}~atoms}{mole} = \mathbf{4.65 \times 10^{20}~atoms~^{239}Pu/g~fuel}$
\item $m(^{240}Pu) =  \frac{0.05769~g}{240.05381375~g/mol}\times \frac{6.022 \times 10^{23}~atoms}{mole} = \mathbf{1.45\times 10^{20}~atoms~^{240}Pu/g~fuel}$
\item $m(^{241}Pu) =  \frac{0.01535~g}{241.056851661~g/mol}\times \frac{6.022 \times 10^{23}~atoms}{mole} = \mathbf{3.83\times 10^{19}~atoms~^{241}Pu/g~fuel}$
\item $m(^{242}Pu) =  \frac{0.007145~g}{242.058742809~g/mol}\times \frac{6.022 \times 10^{23}~atoms}{mole} = \mathbf{1.78\times 10^{19}~atoms~^{242}Pu/g~fuel}$
\end{itemize}

and, finally, we have $1.56 \times 10^{21} + 0.00111~moles~O_{2} \times \frac{6.022 \times 10^{23}~atoms}{mole} = \mathbf{2.23 \times 10^{21}~molecules~O_{2}~per~gram~of~fuel}$.

\hrulefill

\section*{Problem 2}

Determine the mean free path of neutrons for the following energies and materials:

\begin{enumerate}[label=(\alph*)]
\item 14 MeV neutrons in air, water, and uranium—typical of fusion reactors;
\item 1 MeV neutrons in air, water, and uranium—typical of fast reactors;
\item 0.05 eV neutrons in air, water, and uranium—typical of thermal reactors.
\end{enumerate}

Use STP for densities of the materials.

\section*{Problem 3}

Plot the fission and capture cross sections of $^{235}U$, $^{238}U$, and $^{239}Pu$. Rank these isotopes in
terms of their capture-to-fission cross section ratio at 0.0253 eV and at 0.73 MeV (the most
probable energy of fission neutrons).

\section*{Problem 4}

Boron and gadolinium are used as neutron absorber in LWRs (Light Water Reactors). By
means of which reaction they absorb neutrons? Plot the corresponding cross sections. Which
of their isotopes is the most absorbing at 0.0253 eV?

\section*{Problem 5}

Determine the fission-rate density necessary to produce a thermal power density of 400 kW/liter
(typical of a fast breeder reactor core). Assume that the principal fissile isotope is $^{239}Pu$.

\section*{Problem 6}

An indium foil is counted at 4:00 p.m. Tuesday and found to yield 346,573 CPM in a counter
with a 50$\%$ efficiency for the 54-min In-116m activity. What is the probability that none of
these radioactive In-116m nuclei will remain in the foil at 2:00 p.m. Thursday, the same week?
Note: $ln (1- x) \simeq x $ when $ x \ll 1$.

\hrulefill


Because the counter has a 50$\%$ efficiency, we use a total count of

$$\frac{346,573~CPM}{50\%} = 693,146~CPM$$

which is the number of Beta-minus decays per minute, or the activity. Activity is equal to

$$\mathit{A} = \frac{ln2}{t_{1/2}}N$$

where $\frac{ln2}{t_{1/2}} = \lambda = 0.012836058899258~min^{-1}$ for a half-life of 54 minutes. Solving for $N$, there are approximately $5.40 \times 10^{7}$ nuclei of In-116m at 4:00 p.m. on Tuesday. The time that will elapse between 4:00 p.m. Tuesday and 2:00 p.m. Thursday is 46 hours or 2760 minutes. From Duderstadt and Hamilton p. 13, the probability that a given nucleus will decay in a time interval $t$ to $t + dt$ is

$$p(t)dt = \lambda e^{-\lambda t} dt$$

and we are looking for the probability that these nuclei decay within the time interval so we must integrate over the interval, taking our starting time as $t = 0$:

$$\int_{0}^{2760~min} p(t) dt = \lambda \int_{0}^{2760~min} e^{- \lambda t} dt$$

which gives a probability of essentially $1$ that any given In-116m nucleus will decay in this interval. However, because the probability is \textit{not quite} $1$, we can carry our calculation forward. We then set this probability to the power of the number of nuclei to find the probability that\textit{all} of these nuclei decay and arrive at our answer of \boxed{0.999999976019183}. (All calculations in MATLAB)

\hrulefill


\section*{Problem 7}

Compute and plot the parameter $\eta$ for uranium as a function of its enrichment in $^{235}U$ at
thermal neutron energies. Which is the minimum theoretical enrichment value required for
criticality? And which for breeding?

\hrulefill

$\eta$ is the neutron reproduction factor and it is defined as the average number of neutrons produced per neutron absorbed in the fuel. For a mixture of isotopes, it is defined as (Duderstadt and Hamilton p.68):

$$\eta = \frac{\sum_{j} \nu_{j}\Sigma_{f}^{j}}{\sum_{j} \Sigma_{a}^{j}}$$

where, for uranium fuel, we neglect $^{234}U$ and assume it consists of $^{235}U$ and $^{238}U$ alone. For our case, we assume we have a uniform average neutrons per fission of $2.4$ and have the following equation:

$$\eta = \frac{\nu (\Sigma_{f}^{235} + \Sigma_{f}^{238})}{\Sigma_{a}^{235} + \Sigma_{a}^{238}}$$

and macroscopic cross section, $\Sigma = \sigma N$. Using \textit{atom.kaeri.re.kr} and ENDF/B-VII.1 for cross-section data, at thermal energies, we have:

$$\sigma_{f}^{238} = 1.67946 \times 10^{-5}~b~,~\sigma_{\gamma}^{238} = 2.68261~b~,~\sigma_{f}^{235} = 585.086~b~,~\sigma_{\gamma}^{235} = 98.6864~b$$

Then, because absorption includes both fission and absorption, we can write our equation:

$$\eta = \frac{\nu (N^{235}\sigma_{f}^{235} + N^{238}\sigma_{f}^{238})}{N^{235}(\sigma_{f}^{235} + \sigma_{\gamma}^{235}) + N^{238}(\sigma_{f}^{238} + \sigma_{\gamma}^{238})}$$

Here, because $N^{238}$ and $N^{235}$ will each be some fraction of a total $N$, we can use these fractions in place of actual numbers of nuclei. We then have the constraint that they will add to $1$. Plugging this information into MATLAB, we can generate the desired plot:

\begin{figure}[h]
    \centering
    \includegraphics[width=0.7\linewidth]{P7.png}
    \caption{$\eta$ as a function of uranium enrichment at thermal neutron energies}
\end{figure}

Using the data in MATLAB that is illustrated on this plot, we find that the minimum enrichment for criticality, where $\eta \geq 1$, is \boxed{at\%~ ^{235}U = 0.38 \%} and the minimum enrichment for breeding, where $\eta > 2$, is \boxed{at\%~^{235}U = 12.77 \%}.

\hrulefill

\section*{Problem 8}

What is the maximum theoretical value for the multiplication factor in any system? Explain.

\hrulefill

Because the multiplication factor, $k$, is defined as

$$k = \frac{neutrons~in~next~generation}{neutrons~in~this~generation}$$

we know that our numerator is some multiplication of our denominator. The maximum number of neutrons that we can have available in the next generation is the number of neutrons from fission, $\nu$. Therefore, in the ideal case,

$$k = \frac{\nu N}{N}$$

which means, for any system,

$$\boxed{k_{max} = \nu}$$

\hrulefill

\section*{Problem 9}

A spherical reactor composed of $^{235}U$ metal is operating in a critical steady state. Discuss
what probably happens to the multiplication of the reactor and why, if the system is modified
in the following ways (treat each modification separately, not cumulatively):
\begin{enumerate}[label=(\alph*)]
\item the reactor is rapidly compressed to one-half its original volume;
\item the reactor is squashed into an ellipsoidal shape;
\item a thick sheet of cadmium is wrapped around the outside of the reactor;
\item the reactor is suddenly immersed in a large container of water;
\item a source of neutrons is placed near the reactor;
\item another identical reactor is placed a short distance from the original reactor;
\item one simply leaves the reactor alone for a period of time.
\end{enumerate}

\section*{Problem 10}

What are the underlying assumptions in the neutron transport equation? Provide at least five.

\section*{Problem 11}

An interesting model of neutron transport involves transport in a one-dimensional rod. This
means neutrons can move only to the left and or to the right. The angular density of the
neutrons moving to the left and to the right are indicated as $n_{-}(x,E,t)$ and $n_{+}(x,E,t)$,
respectively. One need only to consider forward $\Sigma^{+}_{s}
(E' \rightarrow E)$ or backward $\Sigma^{-}_{s}(E' \rightarrow E)$ scattering events.

\begin{enumerate}[label=(\alph*)]
\item Derive the transport equation for $n_{-}(x,E, t)$ and $n_{+}(x,E,t)$.
\item Make the one-speed approximation, assuming cross sections are energy independent and
in particular $\Sigma^{\pm}_{s}(E' \rightarrow E) = \Sigma^{\pm}_{s} \delta (E' \rightarrow E)$.
\item Describe the boundary and initial conditions necessary to complete the problem specification
assuming the rod length is $L$.
\end{enumerate}

\section*{Problem 12}

Suppose the angular flux in a slab geometry is given by
$$\phi(z,\mu) = \phi_{0}(cosBz + A\mu sinBz)$$
where $\mu = cos\theta$, and $\theta$ measures the angular deviation form the $z$ axis.
(a) Find $\phi(z)$ and $J(z)$.
(b) Rewrite $\phi(z,\mu)$ as a function of $\phi(z)$ and $J(z)$ by eliminating $A$.
(c) Find partial current $J^{+}$ and $J^{-}$ in the upper and lower half of the solid angle.
(d) Check the results for $J^{+}$ and $J^{-}$, comparing them with $J$.

\section*{Problem 13}

Show that in diffusion theory positive/negative partial current through a surface $S$ is given by
the following $(\forall \vec{r} \epsilon S)$:
$$J_{\pm}(\vec{r},t) \simeq \frac{1}{4}\phi(\vec{r},t) \mp \frac{D(\vec{r})}{2}\hat{e}_{s}(\vec{r}) \cdot \nabla \phi(\vec{r},t) $$

\end{document}