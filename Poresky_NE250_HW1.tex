\documentclass{article}
\usepackage[version=3]{mhchem}
\usepackage[english]{}
\usepackage[margin=2cm]{geometry}
\usepackage{enumitem}
\usepackage{graphicx}
\usepackage{float}
\usepackage{amsmath}
\usepackage{dcolumn}
\usepackage{bm}
\usepackage{amssymb}
\usepackage{verbatim}
\usepackage{color}
\definecolor{mygreen}{RGB}{28,172,0} % color values Red, Green, Blue
\definecolor{mylilas}{RGB}{170,55,241}
\usepackage{setspace}
\usepackage{parskip}
\usepackage[space]{cite}
\usepackage{url}
\usepackage{subcaption}
\usepackage{textcomp}
\usepackage{multirow}% http://ctan.org/pkg/multirow
\usepackage{hhline}% http://ctan.org/pkg/hhline
\usepackage{icomma}
\usepackage{placeins}
\usepackage[caption=false,font=footnotesize]{subfig}
% define the title
\author{\textit {Christopher Poresky}}
\title{\textbf {NE 250: Homework 1}}
\begin{document}

\maketitle
\section*{Problem 1}

The fuel for a certain breeder reactor consists of pellets composed of mixed oxides, $UO_{2}$ and
$PuO_{2}$, with the $PuO_{2}$ comprising approximately 30 weight$~\%$ of the mixture. The uranium is
essentially all $^{238}U$, whereas the plutonium contains the following isotopes: $^{239}Pu~(69.7~wt\%)$,
$^{240}Pu~(21.8~wt\%)$, $^{241}Pu~(5.8~wt\%)$, and $^{242}Pu~(2.7~wt\%)$. Calculate the number of atoms of
each isotope per gram of the fuel.

\section*{Problem 2}

Determine the mean free path of neutrons for the following energies and materials:

\begin{enumerate}[label=(\alph*)]
\item 14 MeV neutrons in air, water, and uranium—typical of fusion reactors;
\item 1 MeV neutrons in air, water, and uranium—typical of fast reactors;
\item 0.05 eV neutrons in air, water, and uranium—typical of thermal reactors.
\end{enumerate}

Use STP for densities of the materials.

\section*{Problem 3}

Plot the fission and capture cross sections of $^{235}U$, $^{238}U$, and $^{239}Pu$. Rank these isotopes in
terms of their capture-to-fission cross section ratio at 0.0253 eV and at 0.73 MeV (the most
probable energy of fission neutrons).

\section*{Problem 4}

Boron and gadolinium are used as neutron absorber in LWRs (Light Water Reactors). By
means of which reaction they absorb neutrons? Plot the corresponding cross sections. Which
of their isotopes is the most absorbing at 0.0253 eV?

\section*{Problem 5}

Determine the fission-rate density necessary to produce a thermal power density of 400 kW/liter
(typical of a fast breeder reactor core). Assume that the principal fissile isotope is $^{239}Pu$.

\section*{Problem 6}

An indium foil is counted at 4:00 p.m. Tuesday and found to yield 346,573 CPM in a counter
with a 50$\%$ efficiency for the 54-min In-116m activity. What is the probability that none of
these radioactive In-116m nuclei will remain in the foil at 2:00 p.m. Thursday, the same week?
Note: $ln (1- x) \simeq x $ when $ x \ll 1$.


\section*{Problem 7}

Compute and plot the parameter $\eta$ for uranium as a function of its enrichment in $^{235}U$ at
thermal neutron energies. Which is the minimum theoretical enrichment value required for
criticality? And which for breeding?

\section*{Problem 8}

What is the maximum theoretical value for the multiplication factor in any system? Explain.

\section*{Problem 9}

A spherical reactor composed of $^{235}U$ metal is operating in a critical steady state. Discuss
what probably happens to the multiplication of the reactor and why, if the system is modified
in the following ways (treat each modification separately, not cumulatively):
\begin{enumerate}[label=(\alph*)]
\item the reactor is rapidly compressed to one-half its original volume;
\item the reactor is squashed into an ellipsoidal shape;
\item a thick sheet of cadmium is wrapped around the outside of the reactor;
\item the reactor is suddenly immersed in a large container of water;
\item a source of neutrons is placed near the reactor;
\item another identical reactor is placed a short distance from the original reactor;
\item one simply leaves the reactor alone for a period of time.
\end{enumerate}

\section*{Problem 10}

What are the underlying assumptions in the neutron transport equation? Provide at least five.

\section*{Problem 11}

An interesting model of neutron transport involves transport in a one-dimensional rod. This
means neutrons can move only to the left and or to the right. The angular density of the
neutrons moving to the left and to the right are indicated as $n_{-}(x,E,t)$ and $n_{+}(x,E,t)$,
respectively. One need only to consider forward $\Sigma^{+}_{s}
(E' \rightarrow E)$ or backward $\Sigma^{-}_{s}(E' \rightarrow E)$ scattering events.

\begin{enumerate}[label=(\alph*)]
\item Derive the transport equation for $n_{-}(x,E, t)$ and $n_{+}(x,E,t)$.
\item Make the one-speed approximation, assuming cross sections are energy independent and
in particular $\Sigma^{\pm}_{s}(E' \rightarrow E) = \Sigma^{\pm}_{s} \delta (E' \rightarrow E)$.
\item Describe the boundary and initial conditions necessary to complete the problem specification
assuming the rod length is $L$.
\end{enumerate}

\section*{Problem 12}

Suppose the angular flux in a slab geometry is given by
$$\phi(z,\mu) = \phi_{0}(cosBz + A\mu sinBz)$$
where $\mu = cos\theta$, and $\theta$ measures the angular deviation form the $z$ axis.
(a) Find $\phi(z)$ and $J(z)$.
(b) Rewrite $\phi(z,\mu)$ as a function of $\phi(z)$ and $J(z)$ by eliminating $A$.
(c) Find partial current $J^{+}$ and $J^{-}$ in the upper and lower half of the solid angle.
(d) Check the results for $J^{+}$ and $J^{-}$, comparing them with $J$.

\section*{Problem 13}

Show that in diffusion theory positive/negative partial current through a surface $S$ is given by
the following $(\forall \vec{r} \epsilon S)$:
$$J_{\pm}(\vec{r},t) \simeq \frac{1}{4}\phi(\vec{r},t) \mp \frac{D(\vec{r})}{2}\hat{e}_{s}(\vec{r}) \cdot \nabla \phi(\vec{r},t) $$

\end{document}